\chapter{Inleiding}
\label{inleiding}
Configuratiebeheergereedschappen laten toe op een effici\"ente manier IT infrastructuren op te zetten en onderhouden.
Daarvoor gebruiken ze een model dat de gewenste toestand van de volledige configuratie beschrijft.
%TODO: voorbeeld van zo'n configuratie?
De configuratie bestaat uit een reeks machines, elk met de bestanden, mappen, services,\ldots die ze moeten aanbieden.
Bij het uitrollen van een dergelijke configuratie inspecteert de tool de huidige toestand van elke machine en vergelijkt ze met de gewenste
toestand.
Als er een verschil is maakt de tool de nodige aanpassingen. Indien niet onderneemt de tool geen actie.
Om de vergelijking uit te voeren moet eerst het model gecompileerd.
De huidige tools compileren voor elke machine hun deel van het model.
Dit zorgt ervoor dat bij de vergelijking de tool enkel informatie over die ene machine kan gebruiken.
Daardoor gebeurt het soms dat er teveel gecorrigeerd wordt en de gewenste toestand nog steeds niet bereikt is.
%TODO hoe vaak is soms? Wat triggert "soms"?
De toestand is wel al minder afwijkend van de gewenste situatie.
Een tool zal nooit aanpassingen maken die zorgen voor een configuratie die verder afwijkt van het model dan voorheen.
Na een paar iteraties zal dus altijd de gewenste configuratie bereikt worden.
%Ander woord voor tool? CMS? -> Definieren
Een grafische voorstelling van dit proces te zien op figuur \ref{fig:correctieAfwijking}.

\begin{figure}[p]
    \centering
    \includegraphics[width=0.8\textwidth]{images/correctieAfwijking.svg}
    \caption{Grafische voorstelling van de iteratief verkleinende afwijking van de gewenste situatie bij het uitrollen van een model.}
    \label{fig:correctieAfwijking}
\end{figure}

Het CMS dat aan DistriNet ontwikkeld wordt, IMP, pakt het probleem op een iets andere manier aan.
%"Het probleem" is nog niet gedefinieerd.


%%% Local Variables: 
%%% mode: latex
%%% TeX-master: "masterproef"
%%% End: 
