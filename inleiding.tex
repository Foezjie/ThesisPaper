\chapter{Inleiding}
\label{inleiding}
Configuratiebeheergereedschappen (of CMS: Configuration Management Software) zoals IMP, Puppet, CFEngine,\ldots laten toe op een effici\"ente manier IT infrastructuren op te zetten en onderhouden.
%Referenties naar sites van hoofdpaginas van CMSen
De gebruiker van een dergelijke tool specifi\"eert eerst een model dat de gewenste toestand van de volledige infrastructuur beschrijft.
Dit model bestaat uit een oplijsting van machines met de gewenste aanwezige resources (bestanden, mappen, services,\ldots) die ze moeten aanbieden.
%Dit model is zo bij de huidige tools, niet echt bij IMP
Een eenvoudige voorbeeldconfiguratie is de LAMP-stack: 
\begin{itemize}
  \item Linux
  \item Apache webserver
  \item MySQL database (of MariaDB, MongoDB,\ldots)
  \item PHP (of Perl, Python,\ldots)
\end{itemize}
%IMP code al tonen?

% Juiste accolades
Bij het uitrollen van een configuratie (een "deployment run") inspecteert de tool de huidige toestand van elke machine en vergelijkt ze met de gewenste toestand.
Als er een verschil is maakt de tool de nodige aanpassingen, indien niet onderneemt de tool geen actie.
Om de vergelijking uit te voeren moet eerst het model gecompileerd worden.
De huidige tools compileren voor elke machine hun deel van het model.
Dit zorgt ervoor dat bij de vergelijking de tool enkel informatie over die ene machine kan gebruiken.
Daardoor gebeurt het soms dat er teveel gecorrigeerd wordt en de gewenste toestand nog steeds niet bereikt is.
We kunnen twee gevallen onderscheiden: modellen met afhankelijkheden tussen resources binnen dezelfde machine en met afhankelijkheden tussen verschillende machines.
Binnen dezelfde machine kan het fout gaan als de afhankelijkheden niet expliciet vermeld staan in het model. 
%Grafiek van file <- parent dir?
%vm1 = ip::Host(name = "server", os = "fedora-18", ip = "172.16.34.14")
%dir = std::Directory(host=vm1, path = "/tmp/dir", owner="root", group = "root", mode = 644)
%file = std::File(host=vm1, path = "/tmp/dir/file", owner="root", group = "root", mode = 644, content = "inhoud", requires = host2file)
Als in het model bijvoorbeeld een bestand en de map waarin het moet staan vermeld zijn, is het mogelijk dat de CMS eerst het bestand zal
proberen te cree\"eren en dan pas de map waarin het moet staan, wat zou leiden tot een foutmelding. De map wordt wel correct aangemaakt.
Pas in de volgende deployment run zullen zowel het bestand als de map aanwezig zijn.
Dit probleem kan vermeden worden door in het model de vermelden dat het bestand afhankelijk is van de map.
Dan zal de CMS weten dat de map eerst moet gemaakt worden en dan pas het bestand.

In vergelijking met de beginsituatie is de toestand van de machine na \'e\'en run al minder afwijkend van de gewenste situatie.
Een tool zal nooit aanpassingen maken die zorgen voor een configuratie die verder afwijkt van het model dan voorheen.
Na een paar iteraties zal dus altijd de gewenste configuratie bereikt worden.
Een grafische voorstelling van dit proces is te zien op figuur \ref{fig:correctieAfwijking}.
%Verwijzen naar een hoofdstuk over convergentie?

\begin{figure}[p]
    \centering
    \includegraphics[width=0.8\textwidth]{images/correctieAfwijking.pdf}
    \caption{Grafische voorstelling van de iteratief verkleinende afwijking van de gewenste situatie bij het uitrollen van een model.}
    \label{fig:correctieAfwijking}
\end{figure}

Een voorbeeld van afhankelijkheden tussen verschillende machines is de relatie tussen de webserver en een databaseserver binnen de daarnet
vermelde LAMP-stack.
Het is wenselijk dat eerst de database online is en dan pas de webserver.
Anders is het mogelijk dat de webserver data opvraagt van een database die nog niet beschikbaar is.
Tussen machines onderling is het met de huidige tools nog niet mogelijk om afhankelijkheden tussen verzamelingen resources (bijvoorbeeld een webserver) op verschillende machines te specifi\"eren. 
De oplossing die er was voor de configuratie binnen \'e\'en host kan dus niet gebruikt worden.
%Vermelden dat Puppet met een workaround dit wel ongeveer kan?

Het doel van deze thesis is om in IMP de functionaliteit toe te voegen die het mogelijk maakt afhankelijkheden tussen verzamelingen resources te specifi\"eren.

%TODO: de probleemstelling is nog niet echt gedefinieerd tot nu toe

%%% Local Variables: 
%%% mode: latex
%%% TeX-master: "masterproef"
%%% End: 
