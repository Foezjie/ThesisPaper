\chapter{Besluit}
\label{sec:besluit}

\section{Gerelateerd werk}
verwijzen naar overzichtspaper

IBM

\section{Verder werk}
%Meer use cases%
Een eerste taak is het testen van de heuristieken op enkele andere use cases om zo hun validiteit verder te bevestigen.
Verder kan het bekijken van andere cases ook leiden tot het vinden van nieuwe heuristieken.
Een deel van de heuristieken werden namelijk ge\"introduceerd na onderzoek van de document processing use case. 

%kijken welke subset van deps nu echt nodig is.%
Een tweede onderzoeksvraag die verder kan onderzocht worden is wat de minimale set vereisten is om een model correct uit te rollen.
De huidige heuristieken voegen liever een extra vereiste toe dan \'e\'en te weinig.
Deze aanpak is goed zolang er geen incorrecte of overbodige vereisten worden toegevoegd.
Elke vereiste moet verwerkt worden door IMP en zorgt dus voor een kleine verhoging in de uitvoertijd.
Een minimale set vereisten heeft dus een positief effect op de totale uitroltijd.

%Deletes%
In tegenstelling tot andere CMS laat IMP toe afhankelijkheden en relaties te specifi\"eren.
Deze thesis gebruikt die extra informatie om het uitrolproces te optimaliseren.
Het is aan de creativiteit van de gebruiker om nog andere doeleinden te vinden.
Een mogelijkheid is uitzoeken hoe beter omgegaan kan worden met resources die in \'e\'en versie van het model staan maar niet meer in een volgende versie.
Momenteel stopt alle CMS dan met het onderhoud van die resource. 
In sommige gevallen kan dit ongewenste gevolgen hebben: een database die online blijft maar een niet meer geupdated wordt is een veiligheidsrisico.
De service stoppen als ze niet meer in het model vermeld staat zou hier een betere optie zijn.
Hierbij kan bijvoorbeeld de database resource weten of hij nog in de configuratie van zijn client staat door te kijken naar de relatie die beide hebben.
Dit is maar \'e\'en voorbeeld van wat de extra informatie in het model toelaat.

zoeken van cycles in dependencies
