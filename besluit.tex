\chapter{Besluit}
\label{sec:besluit}

\section{Gerelateerd werk}
\todo{verwijzen naar overzichtspaper}
Cloud Resource Orchestration: A Data-Centric Approach: andere aanpak om te vermijden dat inconsistente configuraties bereikt worden.
Datacentrische aanpak (declaratief) itt tot andere tools die heel imperatief werken (bvb ansible)

\todo{IBM}

\section{Verder werk}
%Meer use cases%
Een eerste onderzoeksrichting betreft de heuristieken.
Niet alleen worden deze beter nog in andere use cases gebruikt om hun validiteit verder te bevestigen, nieuwe use cases kunnen ook leiden tot nieuwe heuristieken.

De heuristieken kunnen mogelijks ook gebruikt worden als onderdeel van een intelligente autocompletetool. 
Een dergelijke tool stelt de gebruiker voor om vereisten tussen resources binnen een stack toe te voegen, of suggereert om bepaalde relaties als afhankelijk te specificeren.

%kijken welke subset van deps nu echt nodig is.%
Verder onderzoek kan ook pogen de minimale set vereisten te defini\"eren die nodig is om een model correct uit te rollen.
De heuristieken die voortkomen uit dit onderzoek voegen liever een extra vereiste toe dan geen.
Deze aanpak is goed zolang er geen incorrecte of overbodige vereisten worden toegevoegd, maar elke vereiste moet verwerkt worden door IMP.
Dit zorgt voor een kleine verhoging in de uitvoertijd.
Een minimale set vereisten kan dus een positief effect hebben op de totale uitroltijd.

%Deletes%
Een derde richting die verder onderzocht kan worden is hoe IMP (of CMS in het algemeen) kan omgaan met verschillende tussen opeenvolgende versies van een configuratiemodel.
Momenteel stopt alle CMS dan met het onderhoud van die resource. 
In sommige gevallen kan dit ongewenste gevolgen hebben: een database die online blijft maar een niet meer geupdated wordt is een veiligheidsrisico.
De service stoppen als ze niet meer in het model vermeld staat zou hier een betere optie zijn.
Mogelijks leidt dit zelfs tot andere resources die overbodig worden, zoals een database die enkel gebruikt door een webserver.
Als de webserver uitgeschakeld wordt kan het wenselijk zijn om automatisch ook de databaseserver stop te zetten.
Dit soort acties zijn mogelijk in IMP omdat in het model de relaties staan die aanwezig zijn in de opstelling.
